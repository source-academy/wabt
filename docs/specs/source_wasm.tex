\input source_header.tex
\begin{document}
	%%%%%%%%%%%%%%%%%%%%%%%%%%%%%%%%%%%%%%%%%%%%%%%
	\docheader{2021}{Source}{\S 1}{Martin Henz, Lee Ning Yuan, Daryl Tan}
	%%%%%%%%%%%%%%%%%%%%%%%%%%%%%%%%%%%%%%%%%%%%%%%

% \input source_intro.tex

%\input source_1_bnf.tex

\newpage

% \input source_return

\section{Introduction}

\input wasm_intro.tex

\newpage

\section{Syntax}

\input wasm_syntax.tex

\section{Segments}

Each webassembly program is a \textit{module} consisting of a sequence of segments. A module collects definitions for types, functions, tables, memories and globals. In addition, it can declare imports and exports and provide initialisation in the form of data and element segments, or a start function.

\begin{alignat*}{9}
	&& \textit{module}    &&\quad ::= &\quad && \textbf{\texttt{(}}\ \textbf{\texttt{module}}\  \textit{segment}^*\ \textbf{\texttt{)}}\  && \textrm{module}\\[1mm]
	&& \textit{segment}    &&\quad ::= &\quad && \textit{type}&& \textrm{type segment} \\[1mm]
	&& 					   &&\quad 	|  &\quad && \textit{import}&& \textrm{import segment} \\[1mm]
	&& 					   &&\quad 	|  &\quad && \textit{function}&& \textrm{function segment} \\[1mm]
	&& 					   &&\quad 	|  &\quad && \textit{table}&& \textrm{table segment} \\[1mm]
	&& 					   &&\quad 	|  &\quad && \textit{memory}&& \textrm{memory segment} \\[1mm]
	&& 					   &&\quad 	|  &\quad && \textit{global}&& \textrm{global segment} \\[1mm]
	&& 					   &&\quad 	|  &\quad && \textit{export}&& \textrm{export segment} \\[1mm]
	&& 					   &&\quad 	|  &\quad && \textit{start}&& \textrm{start segment} \\[1mm]
	&& 					   &&\quad 	|  &\quad && \textit{element}&& \textrm{element segment} \\[1mm]
	&& 					   &&\quad 	|  &\quad && \textit{data}&& \textrm{data segment} \\[1mm]
\end{alignat*}

\input wasm_type.tex

\input wasm_import.tex

\input wasm_function.tex

\input wasm_table.tex

\input wasm_element.tex

\input wasm_memory.tex

\input wasm_global.tex

\input wasm_export.tex

\input wasm_start.tex

%\input wasm_data.tex
\end{document}
